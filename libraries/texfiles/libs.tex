\section{Python libraries}
\subsection{Loading libraries}

\begin{frame}[fragile]
\frametitle{Install libraries}
To install libraries, you can either use \href{https://docs.conda.io/en/latest/}{conda}, \href{https://pip.pypa.io/en/stable/}{pip} or install from source files.
\vspace{1em}

\emph{Pip} only installs Python packages. The user needs to manually install the required external tools (compilers, C libraries, etc.).\\
\vspace{1em}

\emph{Conda} installs everything that is needed (compiler, libraries) and has the ability to create isolated environments, which may contains different versions of the packages.
\vspace{1em}

\begin{block}{Pip vs. conda}
For further details about the difference between Pip and Conda, visit \href{https://www.anaconda.com/understanding-conda-and-pip/}{anaconda.com}.
\end{block}
\end{frame}

\begin{frame}[fragile]
\frametitle{Install from Pip}
All the packages available on Pip are listed on the \href{https://pypi.org/}{pip} webpage. To install any of them, type on a terminal:
\begin{lstlisting}[basicstyle=\scriptsize]
pip install package_name
\end{lstlisting}
\vspace{1em}

For install, to manipulate NetCDF files, 
\begin{lstlisting}[basicstyle=\scriptsize]
pip install netCDF4 xarray
\end{lstlisting}

\end{frame}

\begin{frame}[fragile]
\frametitle{Install from Conda}
All the packages available on the Conda default channel are listed on the \href{https://repo.anaconda.com/pkgs/}{Conda repository}. To install any of them, type on a terminal:
\begin{lstlisting}[basicstyle=\ttfamily\scriptsize]
conda install package_name
\end{lstlisting}
\vspace{1em}

For install, to manipulate NetCDF files, 
\begin{lstlisting}[basicstyle=\ttfamily\scriptsize]
conda install netCDF4 xarray
\end{lstlisting}
\end{frame}

\begin{frame}[fragile]
\frametitle{Conda virtual environments}

To create virtual environments (for instance for maps), type in a terminal:

\begin{lstlisting}[language=bash]
# Mapping env. for Python 3
conda create --name pyngl --channel conda-forge pynio pyngl
source activate pyngl

# Mapping env. for Python 2
conda create --name pyngl2 --channel conda-forge pynio pyngl python=2.7
source activate pyngl2
\end{lstlisting}

In this case, the packages are downloaded from a community channel, the \href{https://conda-forge.org/}{conda-forge}.\\
\vspace{1em}

To list all the environments, type: 

\begin{lstlisting}[language=bash, basicstyle=\ttfamily\scriptsize] 
conda env list
\end{lstlisting}

\begin{block}{Note}
The default environment is named \verb+base+
\end{block} 

\end{frame}

\begin{frame}[fragile]
    \frametitle{Install from source}

    To install a package from sources, unzip the archive and type:
    \begin{lstlisting}[language=bash, basicstyle=\ttfamily\scriptsize] 
python setup.py install --home=/my/directory/
    \end{lstlisting}

    \begin{block}{Note}
At the beginning, you are unlikely to do that.
    \end{block}

\end{frame}

\begin{frame}[fragile]
    \frametitle{Loading libraries}
    Libraries are loaded by using the \verb+import+ statement. It can be done as follows:
    \lstinputlisting[linerange={1-7}, basicstyle=\ttfamily\scriptsize]{scripts/load_lib.py}

    \vspace{1em}
    Here, the objects of the imported modules are stored into defined \emph{namespaces}, which prevent from conflicts among object names.
\end{frame}

\begin{frame}[fragile]
    \frametitle{Loading libraries}
    Libraries can also be loaded as follows:
    \lstinputlisting[linerange={9-16}, basicstyle=\ttfamily\scriptsize]{scripts/load_lib.py}
    In this case, part or all the content of the modules is stored in the current namespace.
    \begin{alertblock}{Advice}
I strongly recommend to \textbf{never} use this way of importing modules, since they may be in conflict with other objects.
    \end{alertblock}
\end{frame}

\begin{frame}[fragile]
    \frametitle{Loading libraries}
    \lstinputlisting[linerange={18-100100}, basicstyle=\ttfamily\scriptsize]{scripts/load_lib.py}
    \vspace{1em}
    \scriptsize
    \begin{verbatim}
[4. 6.]
Traceback (most recent call last):
  File "load_lib.py", line 24, in <module>
    print(log10(x))
TypeError: only size-1 arrays can be converted to Python scalars
    \end{verbatim}
    \normalsize
    
    The \verb+log10+ method of \verb+numpy+ has been overwritten by the \verb+math+ one.

\end{frame}

\begin{frame}[fragile]
\frametitle{Loading your libraries}
    If you want to load your own function, stored in a \verb+mylib.py+ file, you first need to add the directory to the \verb+PYTHONPATH+:
    \begin{lstlisting}[basicstyle=\ttfamily\scriptsize]
import sys
sys.path.append('/add/other/directory/')
import mylib
\end{lstlisting}

\begin{block}{Hint}
    \small
You can create the PYTHONPATH environment variable. With Mac Os X/Linux,  edit your \verb+.bashrc+ or \verb+.cshrc+ file and add:
\begin{verbatim}
# bashrc
export PYTHONPATH=${PYTHONPATH}:/add/other/directory
# cshrc
setenv PYTHONPATH /add/other/directory:${PYTHONPATH}
\end{verbatim}
\end{block}
\end{frame}
